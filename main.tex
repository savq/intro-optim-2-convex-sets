%! TEX program = xelatex
%! TEX encoding = UTF-8 Unicode

\documentclass[11pt]{article}
\usepackage{./unal}
\geometry{letterpaper}

\title{Conjuntos Convexos}
\author{
    Sergio Alejandro Vargas |
    \href{mailto:savargasqu@unal.edu.co}{savargasqu@unal.edu.co} \\
    \normalsize{Introducción a la Optimización --- 2022-I}\\
    \normalsize{Universidad Nacional de Colombia}
}
\date{\today}

\setlength{\parindent}{0pt} % don't indent paragraph
\setlength{\parskip}{1em}   % separate them with whitespace

\newenvironment{enunciado}[1]{\textbf{#1}\itshape}{}


% Notation
\newcommand{\R}{\mathbf{R}} % real numbers
\newcommand{\dom}{\mathbf{dom}\,}

\newcommand{\conv}{\mathbf{conv}\,} % convex hull of set
\newcommand{\dist}{\mathbf{dist}}
\renewcommand{\int}{\mathbf{int}\,} % interior of set
\renewcommand{\rank}{\mathbf{rank}\,}
\renewcommand{\ker}{\mathcal{N}}

\newcommand{\E}{\mathbf{E}\,} % expected value
\newcommand{\prob}{\mathbf{prob}}
\newcommand{\quart}{\mathbf{quartile}}
\renewcommand{\var}{\mathbf{var}} % variance


\begin{document}
\maketitle
%! TEX root = ../main.tex

\begin{enunciado}{2.1}
Let $C \subseteq \R^n$ be a convex set, with $x_1, \dots, x_k \in C$,
and let $\theta_1, \dots, \theta_k \in \R$ satisfy $\theta_i \geq 0$,
$\theta_1 + \cdots + \theta_k = 1$. Show that $\theta_1x_1 + \cdots + \theta_kx_k \in C$.
(The definition of convexity is that this holds for $k = 2$; you must show it for arbitrary $k$.)
Hint. Use induction on k.
\end{enunciado}


Por definición sabemos que la afirmación se cumple para $k = 2$.
Ahora, supongamos que $\theta_1x_1 + \cdots + \theta_kx_k \in C$ se cumple para algún $k$ arbitrario.
Veamos si se cumple para $k+1$.

Tomemos \(x_1, \dots x_k, x_{k+1} \in C\), y
$\theta_1, \dots, \theta_k, \theta_{k+1} \in \R$ tal que
$\theta_1 + \cdots + \theta_k + \theta_{k+1} = 1$.
Podemos reescribir la condición de la suma como
\begin{align*}
    \sum_{i=1}^{k+1} \theta_i   &= 1 \\
    \sum_{i=1}^{k} \theta_i     &= 1 - \theta_{k+1} \\
    \frac{1}{1-\theta_{k+1}}\sum_{i=1}^{k} \theta_i &= 1 \\
    \sum_{i=1}^{k} \frac{\theta_i}{1-\theta_{k+1}}  &= 1
\end{align*}
Si definimos $\eta_i = \frac{\theta_i}{1-\theta_{k+1}}$ tenemos $\sum_{i=1}^{k} \eta_i  = 1$.


Así, podemos reescribir la combinación $\theta_1x_1 + \dots \theta_kx_k + \theta_{k+1}x_{k+1}$
como $(1-\theta_{k+1})(\eta_1x_1 + \cdots + \eta_kx_k) + \theta_{k+1}x_{k+1}$.
Por hipótesis, si $x_1, \dots, x_k \in C$ y $\sum \eta_i = 1$ entonces $(\eta_1x_1 + \cdots + \eta_kx_k) \in C$.
Llamemos al resultado de esta combinación $y$.
Como el conjunto $C$ es convexo y $x,y\in C$ entonces $(1-\theta_{k+1})y + \theta_{k+1}x_{k+1} \in C$.
Por el principio de inducción matemática la afirmación se cumple para todo $k$. 



%! TEX root = ../main.tex

\begin{enunciado}{2.3}
Midpoint convexity. A set $C$ is midpoint convex if whenever two points $a, b$ are in $C$, the
average or midpoint $(a + b)/2$ is in $C$. Obviously a convex set is midpoint convex. It can
be proved that under mild conditions midpoint convexity implies convexity. As a simple
case, prove that if $C$ is closed and midpoint convex, then $C$ is convex.
\end{enunciado}

Recordemos que un conjunto $C$ es cerrado si y solo si contiene sus punto límite, es decir,
el límite de cualquier sucesión de puntos de $C$ converge en un punto en $C$.

Sean $a,b \in C$. Como $C$ es punto-medio convexo, entonces $(a + b)/2$.
Definimos la sucesión $s_n = (s_{n-1} + s_{n-2})/2$
Si tomamos $s_0 = x, s_1 = y$ obtenemos:

\begin{align*}
    s_2 = (y + x)/2
    s_3 = (s_3 + y)/2
    \vdots
\end{align*}

Vemos que cada iteración obtendremos un punto más cerca de $y$.

Como todos los puntos de la sucesión pertenecen a $C$, podemos tomar
dos puntos cualesquiera producidos por la sucesión como los
valores iniciales de otra sucesión producida por la misma regla.
Así podemos ver que ``llenar'' el segmento de recta entre $x,y$,
probando que el conjunto es convexo.


%! TEX root = ../main.tex

\begin{enunciado}{2.4}
Show that the convex hull of a set $S$ is the intersection of all convex sets that contain $S$.
\end{enunciado}

Sea $C = \{ A \mid S \subseteq A, \text{tal que $A$ es convexo} \}$.
Por definición $\conv(S)$ contiene a $S$ y es convexa.
entonces $\conv(S)$ es un elemento de $C$,
por lo que $\bigcap\limits_{A\in C} A \subseteq conv(S)$.

Ahora, sea $x \in \conv(S)$.
El punto $x$ debe ser la combinación convexa de puntos $x_1, \dots, x_n$ en $S$.
Si tomamos un conjunto convexo $A$ tal que $S \subseteq A$,
entonces $x_1, \dots, x_n \in A$.
Como $A$ es convexo entonces también se tiene que $x \in A$.
Si $x\in A$ para todo $A \in C$ tenemos que $x\in \bigcap\limits_{A\in C} A$
luego $\conv(S) \subseteq \bigcap\limits_{A\in C} A$.

La envolvente convexa contiene la intersección, y la intersección contiene la envolvente convexa.
Por lo tanto $\conv(S) = \bigcap\limits_{A\in C} A$.


%! TEX root = ../main.tex

\begin{enunciado}{2.5}
What is the distance between two parallel hyperplanes
$\{x \in \R^n \mid a^Tx = b_1\}$ and
$\{x \in \R^n \mid a^Tx = b_2\}$?
\end{enunciado}
% https://math.stackexchange.com/a/1484670

Sea $H_1 = \{x \in \R^n \mid a^Tx = b_1\}$ y sea  $H_2 = \{x \in \R^n \mid a^Tx = b_2\}$.
Sea $x_1$ un punto cualquiera en $H_1$ y sea $L$ la recta que pasa por $x_1$ ortogonal a $H_1$ (paralela a $a$),
es decir, $L = \{x_1+at\}$ para todo $t\in\R$.

Como $L$ es ortogonal a $H_1$, y además $H_2$ es paralelo a $H_1$ entonces $L$ intersecta
a $H_2$ en un punto $x_2 = x_1 + at$ para algún $t$. La distancia entre $x_1$ y $x_2$ es 
la distancia entre los dos hiperplanos.
Para hallar $x_2$, primero hallamos $t$
\begin{align*}
    a^Tx_2 &= b_2 \\
    a^T(x_1 + at) &= b_2 \\
    a^Tx_1 + a^Tat &= b_2 \\
    a^Tat &= b_2 - a^Tx_1 \\
    t &= \frac{b_2 - a^Tx_1}{a^Ta} \\
\end{align*}
Luego $x_2 = x_1 + a\frac{b_2 - a^Tx_1}{a^Ta}$. Y para hallar la distancia vemos que
\begin{align*}
    x_2 &= x_1 + a\frac{b_2 - a^Tx_1}{a^Ta} \\
    x_2 &= x_1 + a\frac{b_2 - b_1}{a^Ta} \\
    x_2 - x_1  &= a\frac{b_2 - b_1}{a^Ta} \\
    \norm{x_2 - x_1}  &= \norm{a\frac{b_2 - b_1}{a^Ta}} \\
    \norm{x_2 - x_1}  &= \norm{a}\frac{\abs{b_2 - b_1}}{a^Ta} \\
    \norm{x_2 - x_1}  &= \norm{a}\frac{\abs{b_2 - b_1}}{\norm{a}^2} \\
    \norm{x_2 - x_1}  &= \frac{\abs{b_2 - b_1}}{\norm{a}}
\end{align*}



%! TEX root = ../main.tex

\begin{enunciado}{2.8}
    2.8 Which of the following sets S are polyhedra?
    If possible, express $S$ in the form $S = \{x | Ax \preceq b, Fx = g\}$.
\end{enunciado}

\begin{enumerate}[label=(\alph*)]
\item $S = \{y_1a_1 + y_2a_2 \mid −1 \leq y_1 \leq 1, −1 \leq y_2 \leq 1\}$, where $a_1,a_2 \in \R^n$.

    Si $a_1, a_2$ son vectores colineales entonces $S$ es una recta, no un poliedro.
    Si no son colineales, entonces $S$ tendrá la forma de un ``paralelogramo'' con esquinas
    \[a_1 + a_2,\quad
        a_1 - a_2,\quad
        -a_1 + a_2,\quad
    -a_1 - a_2\]
    Podemos describir este paralelogramo con el hiper-plano definido por $a_1,a_2$ (de dimensión $n-1$) y
    por cuatro semi-espacios (de dimensión $n-2$) sobre el primer hiper-plano (correspondientes a los lados del ``paralelogramo'').

    El plano definido por $a_1,a_2$ se puede definir por medio de $n-2$ ecuaciones
    de la forma $v_k^Tx$ para $k=1,\dots, n-2$ donde cada $v_k$ es un vector ortogonal a $a_1$ y a $a_2$.
    Llamemos a este plano $P$.

    Los semi-espacios se pueden definir como:
    \begin{align*}
        S_1 &= \{z + y_1 a_1 + y_2 a_2 \mid a^T_1 z = a^T_2 z = 0, y_1 \leq 1     \} \\
        S_2 &= \{z + y_1 a_1 + y_2 a_2 \mid a^T_1 z = a^T_2 z = 0, -1  \leq y_1   \} \\
        S_3 &= \{z + y_1 a_1 + y_2 a_2 \mid a^T_1 z = a^T_2 z = 0, y_2 \leq 1     \} \\
        S_4 &= \{z + y_1 a_1 + y_2 a_2 \mid a^T_1 z = a^T_2 z = 0, -1  \leq y_2   \}
    \end{align*}

    Sea $c_1$ un vector en $P$ y ortogonal a $a_2$. Por ejemplo, podemos tomar
    \[ c_1 = a_1 - \frac{a_1^T a_2}{\norm{a_2}^2} a^2 \]
    Entonces $x\in S_1$ si
    \[ c_1^T x \leq \abs{c_1^T a_1} \]

    Podemos definir otros tres vectores de manera similar para obtener, junto con
    la restricción del plano las siguientes desigualdades:

    \begin{align*}
        v_k^T x  &\leq 0, \text{ para } k = 1,\dots n-2 \\
        -v_k^T x &\leq 0, \text{ para } k = 1,\dots n-2 \\
        c_1^T x  &\leq \abs{c_1^Ta_1} \\
        c_2^T x  &\leq \abs{c_2^Ta_1} \\
        c_3^T x  &\leq \abs{c_3^Ta_2} \\
        c_4^T x  &\leq \abs{c_4^Ta_2}
    \end{align*}

\item $S = \{x \in \R^n \mid x \succeq 0, \mathbf{1}^Tx = 1,
    \sum_{i=1}^n x_i a_i = b_1, \sum_{i=1}^n x_i a^2_i = b_2\}$,
    where $a_1, \dots, a_n \in \R$ and $b_1, b_2 \in \R$.

    $S$ es un poliedro. Ya está definido por una desigualdad y por tres igualdades.

\item $S = \{x \in \R^n \mid x \succeq 0, x^Ty \preceq 1 \text{ for all } y \text{ with } \norm{y}_2 = 1\}$.

    $S$ no es un poliedro. La condición $\norm{y} = 1$ forma una bola unitaria,
    que no se puede definir por un número finito de semi-espacios e hiper-planos.

    % \item $S = \{x \in \R^n \mid x \succeq 0, x^Ty \preceq 1 \text{ for all } y \text{ with } \sum_{i=1}^n \abs{y_i} = 1\}$.

\end{enumerate}


%! TEX root = ../main.tex

\begin{enunciado}{2.9}
    Voronoi sets and polyhedral decomposition. Let $x_0, \dots, x_K \in \R^n$.
    Consider the set of points that are closer (in Euclidean norm) to $x_0$ than the other $x_i$, i.e.,
    $V = \{x \in\R^n \mid \norm{x − x_0}_2 \leq \norm{x − x_i}_2, i = 1,\dots, K\}$.
    $V$ is called the Voronoi region around $x_0$ with respect to $x_1, \dots, x_K$.
\end{enunciado}

\begin{enumerate}[label=(\alph*)]
    \item Show that $V$ is a polyhedron. Express $V$ in the form $V = \{x \mid Ax \preceq b\}$.

        Consideremos un caso particular con dos puntos $x_1$ y $x_2$ en $\R^2$.
        Intuitivamente, podemos ver que las regiones de Voronoi de estos dos puntos están separadas por una linea
        que pasa por el punto medio entre $x_1$ y $x_2$ y
        es ortogonal a la linea que conecta $x_1$ y $x_2$. Generalizando esta idea a
        $K$ puntos en $\R^n$ se tiene que un punto $x$ está en la región de $x_0$ si
        \begin{align*}
            \norm{x - x_0} &\leq \norm{x - x_i} \\
            (x - x_0)^T(x - x_0) &\leq (x - x_i)^T(x - x_i) \\
            x^Tx - 2x_0^Tx + x_0^Tx_0 &\leq x^Tx - 2x_i^Tx + x_i^Tx_i \\
            - 2x_0^Tx + 2x_i^Tx &\leq + x_i^Tx_i - x_0^Tx_0 \\
            2(x_0 - x_i)^Tx &\leq x_i^Tx_i - x_0^Tx_0 \\
        \end{align*}
        Esta desigualdad define un semi-spacio. Si tomamos la desigualdad para cada punto $x_i$
        tenemos un poliedro que podemos expresar como
        \[V = \{x \mid 2\left[\begin{smallmatrix} x_1 - x_0 \\ \vdots \\ x_K - x_0 \end{smallmatrix}\right] \preceq
        \left[\begin{smallmatrix} x_1^Tx_1 - x_0^Tx_0 \\ \vdots \\ x_K^Tx_K - x_0^Tx_0 \end{smallmatrix}\right] \} \]

%%%%%%%%%%%%%%%%%%%%%%%%%%%%%%%%%%%%%%%%%%%%%%%%%%%%%%%%%%%%%%%%%%%%%%%%%%%%%%%%
    \item Conversely, given a polyhedron $P$ with nonempty interior, show how to find $x_0, \dots, x_K$
        so that the polyhedron is the Voronoi region of $x_0$ with respect to $x_1,\dots,x_K$.

        Tomemos un punto $x_0$ tal que $x_0\in P$. Sabemos que cada hiper-plano $H_i = \{x \mid a_i^Tx = b_i\}$
        está a la misma distancia de $x_0$ que de otro punto $x_i$, es decir,
        la distancia de $x_0$ a $x_i$ es 2 veces la distancia de $x_0$ al hiper-plano.
        Si tomamos $x_0$ y le sumamos 2 veces su distancia al plano en dirección del vector normal al plano $a_i$
        entonces tendremos $x_i$.

        \[ x_i = x_0 + 2a_i\frac{\abs{a_i^T + b_i}}{\norm{a_i}} \]


%%%%%%%%%%%%%%%%%%%%%%%%%%%%%%%%%%%%%%%%%%%%%%%%%%%%%%%%%%%%%%%%%%%%%%%%%%%%%%%%
    % \item We can also consider the sets $V_k = \{x \in\R^n \mid \norm{x − x_k}_2 \leq \norm{x − x_i}_2, i\neq k\}$.
    %     The set $V_k$ consists of points in $\R^n$ for which the closest point in the set $\{x_0,\dots,x_K \}$ is $x_k$.

    %     The sets $V_0,\dots, V_K$ give a polyhedral decomposition of $\R^n$.
    %     More precisely, the sets $V_k$ are polyhedra, $\bigcup_{k=0}^{K} V_k = \R^n$, and
    %     $\int V_i \cap \int V_j = \emptyset$ for $i\neq j$, i.e., $V_i$ and $V_j$
    %     intersect at most along a boundary.

    %     Suppose that $P_1,\dots, P_m$ are polyhedra such that $\bigcup_{i=1}^m P_i = \R^n$,
    %     and $\int P_i \cap \int P_j = \emptyset$ for $i\neq j$.
    %     Can this polyhedral decomposition of $\R^n$ be described as the Voronoi regions
    %     generated by an appropriate set of points?

    % TODO: demostrar por contradicción???
\end{enumerate}

%! TEX root = ../main.tex

\begin{enunciado}{2.10}
Solution set of a quadratic inequality.
Let $C \subseteq \R^n$ be the solution set of a quadratic inequality,
$C = \{x \in \R^n \mid x^TAx + b^Tx + c \leq 0\}$,
with $A \in \S^n, b \in \R^n,$ and $c \in R$.
% Are the converses of these statements true?
\end{enunciado}

\newcommand{\hx}{\hat{x}}

\begin{enumerate}[label=(\alph*)]
\item Show that $C$ is convex if $A \succeq 0$.
    Un conjunto es convexo si y solo su su intersección con una linea (conjunto afín)
    arbitraria $L = \{\hx + tv \mid t\in\R\}$ es convexa.

    Tomando $x = \hx + tv$ tenemos
    \begin{align*}
        (\hx + tv)^T A(\hx + tv) + b^T(\hx + tv) + c
        (\hx^T + tv^T)(A\hx + Atv) + b^T\hx + b^Ttv + c \\
        (\hx^TA\hx + \hx^TAtv + tv^TA\hx + tv^TAtv) + b^T\hx + b^Ttv + c \\
        (\hx^TA\hx + 2\hx^TAvt + v^TAvt^2) + b^T\hx + b^Tvt + c &\text{ ($A$ es simétrica)} \\
        v^TAvt^2 + 2\hx^TAvt + b^Tvt + b^T\hx + \hx^TA\hx + c \\
        v^TAvt^2 + (2\hx^TAv + b^Tv)t + b^T\hx + \hx^TA\hx + c
    \end{align*}
    Sea $\alpha = v^TAv, \beta = (2x^TAv + b^Tv)$ y $\gamma = b^Tx + x^TAx + c$.
    Podemos escribir la intersección de $C$ con $L$ como
    \[ C \cap L = {x^T + tv \mid \alpha t^2 + \beta t + \gamma \leq 0}\]
    La intersección es convexa si $\alpha \geq 0$. Si $A$ es semi-definida
    positiva entonces $\alpha = v^TAv \geq 0$ para todo $v$.

    La recíproca no es cierta. Que $C$ sea convexo no implica que $A \succeq 0$.

% TO-DO
% \item Show that the intersection of $C$ and the hyperplane defined by $g^T x + h = 0$
%     (where $g\neq 0$) is convex if $A + \lambda gg^T \succeq 0$ for some $\lambda\in\R$.
\end{enumerate}


%! TEX root = ../main.tex

\begin{enunciado}{2.13}
Conic hull of outer products. Consider the set of rank-k outer products, defined as
$\{XX^T \mid X \in \R^{n \times k}, \rank{X} = k\}$. Describe its conic hull in simple terms.
\end{enunciado}


El producto exterior $XX^T$ tiene que ser una matriz cuadrada semi-definida positiva,
además $\rank XX^T = \rank X^T  = \rank X = k$.

Ahora, sean $A, B$ matrices semi-definidas de rango $k$, se tiene que
$\rank A+B \leq k$.  Sea $v \in\ker(A + B)$, entonces


\[(A + B)v = 0 \iff v^T (A + B)v = 0 \iff v^T Av + v^T Bv = 0\]
Luego también se tiene que
\[v^T Av = 0 \iff Av = 0, v^T Bv = 0 \iff Bv = 0.\]
Por lo que $v\in\ker(A)$ y  $v\in\ker(B)$.
Es decir, la dimensión del núcleo de $A+B$ debe ser mayor o igual a la de los
núcleos de $A$ o $B$. Entonces el rango de $A+B$ no puede ser menor al de
$A$ o al de $B$.

Como la multiplicación por escalar no afecta el rango, y la suma de matrices
semi-definidas no disminuye el rango, tenemos que la envolvente cónica
es el conjunto de matrices semi-definidas  con un rango mayor o igual a $k$.


%! TEX root = ../main.tex

\begin{enunciado}{2.14}
    Expanded and restricted sets. Let $S \subseteq \R^n$, and let $\norm{\cdot}$ be a norm on $R^n$.
\end{enunciado}

\begin{enumerate}[label=(\alph*)]
    \item \textit{For $a \geq 0$ we define $S_a = \{x \mid \dist(x, S) \leq a\}$,
            where $\dist(x, S) = \inf_{y\in S} \norm{x − y}$.
            We refer to $S_a$ as $S$ expanded or extended by $a$.
        Show that if $S$ is convex, then $S_a$ is convex.}

        Sean $x_1, x_2 \in S_a, \theta\in\R$ tal que $0\leq \theta \leq 1$.
        Si probamos que $\dist$ es una función convexa, probamos que $S_a$ es convexa.
        Usando la definición de $\dist$ tenemos
        \[\dist(\theta x_1 + (1-\theta)x_2, S) = \inf_{y\in S}\norm{\theta x_1 + (1-\theta)x_2 - y} \]
        S es convexo, entonces podemos reemplazar $y$
        \begin{align*}
             &= \inf_{y_1,y_2\in S}\norm{\theta x_1 + (1-\theta)x_2 - (\theta y_1 + (1-\theta)y_2)} \\
             &= \inf_{y_1,y_2\in S}\norm{\theta(x_1 - y_1) +  (1-\theta)(x_2 - y_2)} \\
        \end{align*}
        Por desigualdad triangular
        \begin{align*}
            \dist(\theta x_1 + (1-\theta)x_2, S)&\leq \inf_{y_1,y_2\in S}(\theta\norm{x_1 - y_1} +  (1-\theta)\norm{x_2 - y_2}) \\
                                                &\leq \theta\inf_{y_1,y_2\in S}\norm{x_1 - y_1} +  (1-\theta)\inf{\norm{x_2 - y_2}} \\
        \end{align*}
        $x_1$ y $x_2$ están en $S_a$ entonces $\dist(\theta x_1 + (1-\theta)x_2, S) \leq a$.
        La combinación convexa entre $x_1$ y $x_2$ está en $S_a$, por lo tanto $S_a$ es convexa.


    \item \textit{For $a \geq 0$ we define $S_{−a} = \{x \mid B(x, a) \subseteq S\}$, where $B(x, a)$ is the ball
        (in the norm $\norm{\cdot}$), centered at $x$, with radius $a$.
        We refer to $S_{−a}$ as $S$ shrunk or restricted by $a$,
        since $S_{−a}$ consists of all points that are at least a distance $a$ from $\R^n\backslash S$.
        Show that if $S$ is convex, then $S_{−a}$ is convex.}

        Tomemos dos puntos $x_1, x_2 \in S_{-a}$ y $\theta \in \R$ tal que $1\leq\theta\leq 1$.
        Para todo $u$ tal que $\norm{u} \leq a$ 
        se tiene que $x_1 +u \in S$ y que $x_2 + u \in S$.
        Como S es convexo, si sumamos $u$ a la combinación convexa de $x_1$ con $x_2$ se tiene
        \[ \theta x_1 +(1−\theta )x_2 + u = \theta (x_1 + u)+(1−\theta)(x_2 + u) \in S \]
        Por lo tanto  $\theta x_1 + (1 - \theta) x_2 \in s_a$.
\end{enumerate}


% \input{text/15} % TODO

%! TEX root = ../main.tex

\begin{enunciado}{2.16}
    Show that if $S_1$ and $S_2$ are convex sets in $\R^{m+n}$, then so is their partial sum
    \[
        S = \{(x, y_1 + y_2) \mid x \in \R^m, y_1, y_2 \in \R^n, (x, y_1) \in S_1, (x, y_2) \in S_2\}
    \]
\end{enunciado}

Sean $(x_1, y_1) \in S_1, (x_1, y_2) \in S (x_2,z_1) \in S_1$ y $(x_2, z_2) \in S$.
Consideremos los puntos $(x_1, y_1 + y_2), (x_2, z_1, z_2)$.
Para $0\leq \theta \leq 1$ se tiene
\[
    \theta(x_1, y_1 + y_2) + (1 - \theta)(x_2, z_1 + z_2) =
        (\theta x_1 + (1-\theta)x_1, (\theta y_1 + (1-\theta)y_1) + (\theta z_1 + (1-\theta)z_1)
\]
$S_1$ es convexo entonces $(\theta x_1 + (1-\theta)x_1, (\theta y_1 + (1-\theta)y_1) \in S_1$.
Igualmente $S_2$ es convexo entonces $(\theta x_1 + (1-\theta)x_1, (\theta z_1 + (1-\theta)z_1) \in S_2$.
Luego $\theta(x_1, y_1 + y_2) + (1 - \theta)(x_2, z_1 + z_2) \in S$.
Por lo tanto $S$ es convexo.

 % DONE
% \input{text/17} % TODO
% \input{text/18} % TODO

\end{document}
