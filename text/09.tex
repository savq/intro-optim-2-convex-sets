%! TEX root = ../main.tex

\begin{enunciado}{2.9}
    Voronoi sets and polyhedral decomposition. Let $x_0, \dots, x_K \in \R^n$.
    Consider the set of points that are closer (in Euclidean norm) to $x_0$ than the other $x_i$, i.e.,
    $V = \{x \in\R^n \mid \norm{x − x_0}_2 \leq \norm{x − x_i}_2, i = 1,\dots, K\}$.
    $V$ is called the Voronoi region around $x_0$ with respect to $x_1, \dots, x_K$.
\end{enunciado}

\begin{enumerate}[label=(\alph*)]
    \item Show that $V$ is a polyhedron. Express $V$ in the form $V = \{x \mid Ax \preceq b\}$.

        Consideremos un caso particular con dos puntos $x_1$ y $x_2$ en $\R^2$.
        Intuitivamente, podemos ver que las regiones de Voronoi de estos dos puntos están separadas por una linea
        que pasa por el punto medio entre $x_1$ y $x_2$ y
        es ortogonal a la linea que conecta $x_1$ y $x_2$. Generalizando esta idea a
        $K$ puntos en $\R^n$ se tiene que un punto $x$ está en la región de $x_0$ si
        \begin{align*}
            \norm{x - x_0} &\leq \norm{x - x_i} \\
            (x - x_0)^T(x - x_0) &\leq (x - x_i)^T(x - x_i) \\
            x^Tx - 2x_0^Tx + x_0^Tx_0 &\leq x^Tx - 2x_i^Tx + x_i^Tx_i \\
            - 2x_0^Tx + 2x_i^Tx &\leq + x_i^Tx_i - x_0^Tx_0 \\
            2(x_0 - x_i)^Tx &\leq x_i^Tx_i - x_0^Tx_0 \\
        \end{align*}
        Esta desigualdad define un semi-spacio. Si tomamos la desigualdad para cada punto $x_i$
        tenemos un poliedro que podemos expresar como
        \[V = \{x \mid 2\left[\begin{smallmatrix} x_1 - x_0 \\ \vdots \\ x_K - x_0 \end{smallmatrix}\right] \preceq
        \left[\begin{smallmatrix} x_1^Tx_1 - x_0^Tx_0 \\ \vdots \\ x_K^Tx_K - x_0^Tx_0 \end{smallmatrix}\right] \} \]

%%%%%%%%%%%%%%%%%%%%%%%%%%%%%%%%%%%%%%%%%%%%%%%%%%%%%%%%%%%%%%%%%%%%%%%%%%%%%%%%
    \item Conversely, given a polyhedron $P$ with nonempty interior, show how to find $x_0, \dots, x_K$
        so that the polyhedron is the Voronoi region of $x_0$ with respect to $x_1,\dots,x_K$.

        Tomemos un punto $x_0$ tal que $x_0\in P$. Sabemos que cada hiper-plano $H_i = \{x \mid a_i^Tx = b_i\}$
        está a la misma distancia de $x_0$ que de otro punto $x_i$, es decir,
        la distancia de $x_0$ a $x_i$ es 2 veces la distancia de $x_0$ al hiper-plano.
        Si tomamos $x_0$ y le sumamos 2 veces su distancia al plano en dirección del vector normal al plano $a_i$
        entonces tendremos $x_i$.

        \[ x_i = x_0 + 2a_i\frac{\abs{a_i^T + b_i}}{\norm{a_i}} \]


%%%%%%%%%%%%%%%%%%%%%%%%%%%%%%%%%%%%%%%%%%%%%%%%%%%%%%%%%%%%%%%%%%%%%%%%%%%%%%%%
    % \item We can also consider the sets $V_k = \{x \in\R^n \mid \norm{x − x_k}_2 \leq \norm{x − x_i}_2, i\neq k\}$.
    %     The set $V_k$ consists of points in $\R^n$ for which the closest point in the set $\{x_0,\dots,x_K \}$ is $x_k$.

    %     The sets $V_0,\dots, V_K$ give a polyhedral decomposition of $\R^n$.
    %     More precisely, the sets $V_k$ are polyhedra, $\bigcup_{k=0}^{K} V_k = \R^n$, and
    %     $\int V_i \cap \int V_j = \emptyset$ for $i\neq j$, i.e., $V_i$ and $V_j$
    %     intersect at most along a boundary.

    %     Suppose that $P_1,\dots, P_m$ are polyhedra such that $\bigcup_{i=1}^m P_i = \R^n$,
    %     and $\int P_i \cap \int P_j = \emptyset$ for $i\neq j$.
    %     Can this polyhedral decomposition of $\R^n$ be described as the Voronoi regions
    %     generated by an appropriate set of points?

    % TODO: demostrar por contradicción???
\end{enumerate}
