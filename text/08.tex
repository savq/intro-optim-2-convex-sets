%! TEX root = ../main.tex

\begin{enunciado}{2.8}
    2.8 Which of the following sets S are polyhedra?
    If possible, express $S$ in the form $S = \{x | Ax \preceq b, Fx = g\}$.
\end{enunciado}

\begin{enumerate}[label=(\alph*)]
\item $S = \{y_1a_1 + y_2a_2 \mid −1 \leq y_1 \leq 1, −1 \leq y_2 \leq 1\}$, where $a_1,a_2 \in \R^n$.

    Si $a_1, a_2$ son vectores colineales entonces $S$ es una recta, no un poliedro.
    Si no son colineales, entonces $S$ tendrá la forma de un ``paralelogramo'' con esquinas
    \[a_1 + a_2,\quad
        a_1 - a_2,\quad
        -a_1 + a_2,\quad
    -a_1 - a_2\]
    Podemos describir este paralelogramo con el hiper-plano definido por $a_1,a_2$ (de dimensión $n-1$) y
    por cuatro semi-espacios (de dimensión $n-2$) sobre el primer hiper-plano (correspondientes a los lados del ``paralelogramo'').

    El plano definido por $a_1,a_2$ se puede definir por medio de $n-2$ ecuaciones
    de la forma $v_k^Tx$ para $k=1,\dots, n-2$ donde cada $v_k$ es un vector ortogonal a $a_1$ y a $a_2$.
    Llamemos a este plano $P$.

    Los semi-espacios se pueden definir como:
    \begin{align*}
        S_1 &= \{z + y_1 a_1 + y_2 a_2 \mid a^T_1 z = a^T_2 z = 0, y_1 \leq 1     \} \\
        S_2 &= \{z + y_1 a_1 + y_2 a_2 \mid a^T_1 z = a^T_2 z = 0, -1  \leq y_1   \} \\
        S_3 &= \{z + y_1 a_1 + y_2 a_2 \mid a^T_1 z = a^T_2 z = 0, y_2 \leq 1     \} \\
        S_4 &= \{z + y_1 a_1 + y_2 a_2 \mid a^T_1 z = a^T_2 z = 0, -1  \leq y_2   \}
    \end{align*}

    Sea $c_1$ un vector en $P$ y ortogonal a $a_2$. Por ejemplo, podemos tomar
    \[ c_1 = a_1 - \frac{a_1^T a_2}{\norm{a_2}^2} a^2 \]
    Entonces $x\in S_1$ si
    \[ c_1^T x \leq \abs{c_1^T a_1} \]

    Podemos definir otros tres vectores de manera similar para obtener, junto con
    la restricción del plano las siguientes desigualdades:

    \begin{align*}
        v_k^T x  &\leq 0, \text{ para } k = 1,\dots n-2 \\
        -v_k^T x &\leq 0, \text{ para } k = 1,\dots n-2 \\
        c_1^T x  &\leq \abs{c_1^Ta_1} \\
        c_2^T x  &\leq \abs{c_2^Ta_1} \\
        c_3^T x  &\leq \abs{c_3^Ta_2} \\
        c_4^T x  &\leq \abs{c_4^Ta_2}
    \end{align*}

\item $S = \{x \in \R^n \mid x \succeq 0, \mathbf{1}^Tx = 1,
    \sum_{i=1}^n x_i a_i = b_1, \sum_{i=1}^n x_i a^2_i = b_2\}$,
    where $a_1, \dots, a_n \in \R$ and $b_1, b_2 \in \R$.

    $S$ es un poliedro. Ya está definido por una desigualdad y por tres igualdades.

\item $S = \{x \in \R^n \mid x \succeq 0, x^Ty \preceq 1 \text{ for all } y \text{ with } \norm{y}_2 = 1\}$.

    $S$ no es un poliedro. La condición $\norm{y} = 1$ forma una bola unitaria,
    que no se puede definir por un número finito de semi-espacios e hiper-planos.

    % \item $S = \{x \in \R^n \mid x \succeq 0, x^Ty \preceq 1 \text{ for all } y \text{ with } \sum_{i=1}^n \abs{y_i} = 1\}$.

\end{enumerate}

