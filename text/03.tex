%! TEX root = ../main.tex

\begin{enunciado}{2.3}
Midpoint convexity. A set $C$ is midpoint convex if whenever two points $a, b$ are in $C$, the
average or midpoint $(a + b)/2$ is in $C$. Obviously a convex set is midpoint convex. It can
be proved that under mild conditions midpoint convexity implies convexity. As a simple
case, prove that if $C$ is closed and midpoint convex, then $C$ is convex.
\end{enunciado}

Recordemos que un conjunto $C$ es cerrado si y solo si contiene sus punto límite, es decir,
el límite de cualquier sucesión de puntos de $C$ converge en un punto en $C$.

Sean $a,b \in C$. Como $C$ es punto-medio convexo, entonces $(a + b)/2$.
Definimos la sucesión $s_n = (s_{n-1} + s_{n-2})/2$
Si tomamos $s_0 = x, s_1 = y$ obtenemos:

\begin{align*}
    s_2 = (y + x)/2
    s_3 = (s_3 + y)/2
    \vdots
\end{align*}

Vemos que cada iteración obtendremos un punto más cerca de $y$.

Como todos los puntos de la sucesión pertenecen a $C$, podemos tomar
dos puntos cualesquiera producidos por la sucesión como los
valores iniciales de otra sucesión producida por la misma regla.
Así podemos ver que ``llenar'' el segmento de recta entre $x,y$,
probando que el conjunto es convexo.

